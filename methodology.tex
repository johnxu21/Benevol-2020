\section{Methods and Dataset}
\label{sec:method}
According to the 2020 Stack Overflow Developer Survey\footnote{https://insights.stackoverflow.com/survey/2020} to which over 65,000 developers participated, \js is one the most commonly used programming language (67.7\% of all respondents make use of \js).
%\note{of what?} 
%In addition, the \np package distribution platform was observed to have a higher distribution of package dependencies than other package distributions\cite{Decan:2019:emse}.
%For these reasons we selected \texttt{npm} for our empirical study.\note{I don't get the link between the number of dependencies and the choice of \np. I think, correct me if I'm wrong, the fact that \np is one of the largest collection of (active) packages is enough as a motivation to select it.}
%\tom{It is not even clear to me what is meant by "a higher distribution of package dependencies". What does this mean?}



\textbf{Dataset}. Our dataset of the \gh repositories from the \js ecosystem and their corresponding and \np packages dependents and dependencies were extracted from the \texttt{libraries.io} release 1.6.0 of January 12, 2020. We use the following heuristics to extract the mainline--variant pairs of the software families in the \js ecosystem on \gh: the repository $A$ is a variant of the repository $B$ if (1) the package releases of $A$ and $B$ are distributed on \npm 
%\tom{"repositories" are not distributed on npm, only package releases are. Moreover, there is not necessarily a one-to-one mapping between repositories and package releases} 
and (2) $A$ is a fork--of $B$ on \gh.
%\tom{On \gh?} 
Since we focus on the \js ecosystem on \gh, we extracted software families whose repositories are hosted on \gh and their packages distributed on \npm.
%\tm{This sentence comes much too late and is not sufficiently detailed.} 
For each \textbf{mainline} and its corresponding \textbf{variants} we collected their \textbf{releases}, \textbf{dependencies} and \textbf{dependents} (projects and packages).






