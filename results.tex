\section{Results}
\label{sec:results}
\begin{figure}[htbp]
\vspace{-.3cm}
   \centering
    \includegraphics[scale=0.4]{figures/variants.pdf}
    \caption{Family size (number of variants in a family).}
   
    \label{fig:variants}
\end{figure}

\noindent
$RQ_0$: \textbf{How prevalent are software families?}

With this first RQ, we aim to determine if software families exist in software ecosystems. In the \js ecosystem we discovered a total 10,743 distinct mainlines and 12,813 variants in total. This means that we have a total of 10,743 software families. Figure~\ref{fig:variants} present a histogram of the distribution of the number of variants per mainline. The y-axis (in log-scale) shows the number of mainlines and  the x-axis shows the number of variants per mainline. For example, the first bar tells us that there are 9,280 mainlines that contain only one variant. We also observe two large mainlines containing 16 and 17 variants. The results of $RQ_0$ reveal that software families indeed exist in the \js ecosystem on \gh.%\tm{The last sentence is too generic. It only shows that software families exist in the \np-\gh ecosystem, nothing more.}

\begin{figure}[htbp]
\vspace{-.3cm}
   \centering
    \includegraphics[scale=0.6]{figures/benevolj_releases.pdf}
    \caption{The distribution of the mainline versus variant package releases.}
    \label{fig:releases}
\end{figure}

\tm{It would be nice to have a summary box at the end of each RQ}

$RQ_1$: \textbf{How do the distributions of package dependencies in mainlines and their variants compare to each other?}

With this second RQ, we aim to ascertain if the mainlines and the variants are continuously maintained. 
Figure~\ref{fig:releases} presents a scatter plot showing the distribution of releases for the variants versus the releases for the mainlines. 
On the x-axis we have the number of releases for the mainlines and on the y-axis we have the number of releases for the variants. 
The color of the data points in the graph represent the number cases for the mainlines\,/\,variants. 
For example, the data points on the top left of the graph tell us that there are some variants that have more releases compared to their mainline counterparts. 
This implies the variants are being maintained more than their mainline counterparts.
The data points on the bottom right tell us that there are a number of mainlines having many releases compared to their variant counterparts. 
Overall, we observe more mainlines being maintained compared to their variant counterparts.
However, we also observe a significant amount of variants being maintained. 
This is interesting since developers variants did not make a one off package distribution; they are continuously distributing new releases of their package. 

\begin{figure}[htbp]
\vspace{-.3cm}
   \centering
    \includegraphics[scale=0.6]{figures/benevolj_dependencies.pdf}
    \caption{The distribution of the mainline versus variant package dependencies.}
    \label{fig:dependencies}
\end{figure}

\tm{It would be nice to have a summary box at the end of each RQ}

$RQ_2$: \textit{How do the distributions of package dependencies in mainlines and their variants compare to each other?}

With this RQ, we want to ascertain the frequency of package dependencies on other packages for the mainlines and the variants in the software families. 
Figure~\ref{fig:dependencies} presents a scatter plot showing the distribution of the package dependencies of the variants versus the dependencies of the mainline.
On the x-axis we have the number of dependencies of the mainline. 
On the y-axis we have the number of dependencies of the variant.
The color of the data points in the graph represent the number cases for the mainlines\,/\,variants.
For example, on the top right of the graph we see a few scattered points single case variants telling us that there are a few variants that have many dependencies compared to the mainline counterparts.
We also see many scattered single case mainlines having many dependencies compared to their variant counterparts. 
Overall, we observe the mainlines having more dependencies compared to their variant counterparts.

\begin{figure*}%
    \centering
    \subfloat[Packages]{{\includegraphics[width=8cm]{figures/dependents} }}%
    \qquad
    \subfloat[Projects]{{\includegraphics[width=8cm]{figures/benevolj_projects} }}%
    \caption{Distribution of dependent packages and dependent projects for the mainline versus variants.}%
    \label{fig:packages_and_projects}%
\end{figure*}


\begin{figure*}%
    \centering
    \subfloat[ Mainlines]{{\includegraphics[width=8cm]{figures/benevolj_dependents_mainline.pdf} }}%
    \qquad
    \subfloat[ Variants]{{\includegraphics[width=8cm]{figures/benevolj_dependents_variant.pdf} }}%
    \caption{Distribution of dependent packages versus dependent projects for mainlines and variants.}%
    \label{fig:mainline_variants_packages}%
\end{figure*}

\tm{It would be nice to have a summary box at the end of each RQ}

$RQ_3$: \textbf{The distribution of dependent packages and projects of the mainlines and variants.}\tm{Please phrase each RQ as a real question.}

In this RQ we are interesting in observing if other packages\,/\,projects in the ecosystem depend on the variants.
In Figure~\ref{fig:packages_and_projects} we present the scatter plots showing the distribution of the dependent packages (Figure~\ref{fig:packages_and_projects}-(a)) and dependent projects (Figure~\ref{fig:packages_and_projects}-(b)) for the variants versus mainlines.
The x-axes represent the number of dependent packages for the mainline and number of dependent projects for the variants, respectively.
The y-axes represent the number of dependent packages for the variants and number of dependent projects for the variants, respectively.
The color of the data points in the graph represent the number cases for the mainlines\,/\,variants.
Looking at Figure~\ref{fig:packages_and_projects}-(a), we observe that most of the data points are concentrated on the x-axis. 
This implies that there are very many mainline variants having many dependent packages compared to their variant counterparts.
We observe the same trend for the dependent projects in Figure~\ref{fig:packages_and_projects}-(b).
In both Figure~\ref{fig:packages_and_projects}-(a) and Figure~\ref{fig:packages_and_projects}-(b), we observe that most variants have $<10$ dependent packages\,/\,projects. 
This is still interesting since it implies that some developers do depend on the variants as opposed to their mainlines counterparts offering similar functionality.

\tm{It would be nice to have a summary box at the end of each RQ}
