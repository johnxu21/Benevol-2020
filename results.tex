\section{Results}
\label{sec:results}
\begin{figure}[htbp]
\vspace{-.3cm}
   \centering
    \includegraphics[scale=0.3]{figures/variants.pdf}
    \caption{Family size (number of variants in a family).}
   
    \label{fig:variants}
\end{figure}

\noindent
$RQ_0$: \textbf{How prevalent are software families in the \js ecosystem on \gh?}

With this first RQ, we aim to determine if software families exist in software ecosystems. In the \js ecosystem we discovered a total 10,743 distinct mainlines and 12,813 variants in total. This means that we have a total of 10,743 software families. Figure~\ref{fig:variants} presents a histogram of the distribution of the number of variants per mainline. The y-axis (in log-scale) shows the number of mainlines and  the x-axis shows the number of variants per mainline. For example, the first bar tells us that there are 9,280 mainlines that have only one variant. We also observe two mainlines having 16 and 17 variants. The results of $RQ_0$ reveal that software families indeed exist in the considered ecosystem.%\tm{The last sentence is too generic. It only shows that software families exist in the \np-\gh ecosystem, nothing more.}

\begin{framed}
\noindent
\emph{We have identified 10,743 distinct mainlines and 12,813 variants. This shows that software families exist in the \js ecosystem on \gh.}
\end{framed}

\begin{figure}[htbp]
\vspace{-.3cm}
   \centering
    \includegraphics[scale=0.5]{figures/benevolj_releases.pdf}
    \caption{Distribution of the mainline versus variant package releases.}
    \label{fig:releases}
    %\tm{Wouldn't some kind of heatmap be more appropriate? Just asking.}
\end{figure}

%\tm{It would be nice to have a summary box at the end of each RQ}

$RQ_1$: \textbf{How do the distributions of package dependencies in mainlines and their variants compare to each other?}

%This RQ aims to ascertain if the mainlines and the variants are continuously maintained. 
Figure~\ref{fig:releases} presents a scatter plot of releases for the variants versus the releases for the mainlines. 
The x-axis shows the number of releases for the mainlines and the y-axis shows the number of releases for the variants. 
The color of the data points represents the number of mainline--variant pairs. For example, darkest circle around the point (1,1) tells us that there are over 600 mainline--variant pairs that have one release each in the data point. The lightest circles tell there is only one mainline--variant pair in the data point. For example the point (100, 1) tells us that there is one mainline--fork pair where the mainline has 100 releases and the variant has only one release.
%\tm{Unclear what you mean by number of cases}
%For example, the data points on the top left of the graph tell us that there are some variants that have more releases compared to their mainline counterparts. \tm{This has nothing to do with the "color", hence i do not understand the "for example".}
The data points along the y-axis tell us that there are some mainline--variant pairs where the variants have more releases compared to the mainline. 
This implies that these specific variants are being maintained more than their mainline counterparts.
%The data points on the bottom right tell us that there are a number of mainlines having many releases compared to their variant counterparts. 
Overall, we observe more mainlines being maintained compared to their variant counterparts.
However, we also observe a significant amount of variants being maintained. 
This is interesting since developers variants did not make a one off package distribution; they are continuously distributing new releases of their package. 
%\tm{I have difficulties to interpret the figure and what we can actually observe from it. Not much if you ask me.}\az{It seems that there is no correlation between the number of releases for variants and mainlines. This may suggest that: If the mainline keeps releasing new versions then the variant will release less often. But if the mainline stopped (frequently) releasing new versions then more releases will come from the variant.}

\begin{framed}
\noindent
\emph{While many mainlines are being maintained more that their variant counterparts (which is not surprising), we observe a good number of variants being maintained in parallel. Interestingly we also observed a good number of variants that are more actively maintained than their mainline counterparts.}
\end{framed}
%\tm{Do not use the term "significant" since you have not shown any statistical significance, you did not even conduct any statisitcal test...}

\begin{figure}[htbp]
\vspace{-.3cm}
   \centering
    \includegraphics[scale=0.5]{figures/benevolj_dependencies.pdf}
    \caption{The distribution of the mainline versus variant package dependencies.}
    \label{fig:dependencies}
\end{figure}
%\az{If we need space, we can reduce the size of the figures}

%\tm{It would be nice to have a summary box at the end of each RQ}

$RQ_2$: \textbf{How do the distributions of package dependencies in mainlines and their variants compare to each other?}

%This RQ studies the frequency of package dependencies on other packages for the mainlines and the variants in the software families. 
Figure~\ref{fig:dependencies} presents a scatter plot showing the the package dependencies of the variant versus the dependencies of the mainline.
The x-axis shows the number of dependencies of the mainline and
the y-axis shows the number of dependencies of the variant.
The color of the data points represents the number mainline--variant pairs on the data point.
For example, the darker data points indicate more mainline--variant pairs in the same data point and the lighter data points indicate fewer mainline--variant pairs in the data point. 
We observe some kind of correlation between the dependencies of the mainlines and variants. 
This means the more dependencies mainlines are associated with more dependencies of the variants. This could imply that the fork variant inherits the original dependencies of the mainline.

%on the top right of the graph we see a few scattered points single case variants telling us that there are a few variants that have many dependencies compared to the mainline counterparts.\tm{Same comments above as for previous RQ}
%We also see many scattered single case mainlines having many dependencies compared to their variant counterparts. 
%Overall, we observe the mainlines having more dependencies compared to their variant counterparts.
%\az{I see some kind of correlation which was not the case in Figure~\ref{fig:releases}. This may mean that variants do not change a lot of their dependencies inherited from the mainline.}

\begin{framed} 
\noindent
\emph{We observe a correlation between the number of mainline and variant dependencies. This could imply that the variant continues using the packages inherited from the mainline.}
\end{framed}
%\tm{Do not use the term "significant" since you have not shown any statistical significance, you did not even conduct any statisitcal test...}

\begin{figure}[htbp]
\vspace{-.3cm}
   \centering
    \includegraphics[scale=0.5]{figures/dependents.pdf}
    \caption{The distribution of the dependent packages for mainline versus variant .}
    \label{fig:depsPackages}
\end{figure}

\begin{figure}[htbp]
\vspace{-.3cm}
   \centering
    \includegraphics[scale=0.5]{figures/benevolj_projects.pdf}
    \caption{The distribution of the dependent projects for the mainline versus variant.}
    \label{fig:depsProjects}
\end{figure}

%\tm{It would be nice to have a summary box at the end of each RQ}

$RQ_3$: \textbf{Do the variant projects have dependent packages\,/\,projects?}

%This RQ studies if other packages\,/\,projects in the ecosystem depend on the variants.
Figure~\ref{fig:depsPackages} and~\ref{fig:depsProjects} present a scatter plots for mainlines versus variants for the dependent packages and dependent projects, respectively.
The x-axes represent the number of dependent packages\,/\,projects for the mainlines and the y-axes represent the number of dependent packages\,/\,projects for the variants.
The color of the data points represents the number of mainline--variant pairs at a data point.
For example, in Figure~\ref{fig:depsPackages} the darker colors indicate that we have more mainline--fork pairs on one data point, while the lighter data points indicate the reverse.

Looking at both Figure~\ref{fig:depsPackages} and~\ref{fig:depsProjects}, we observe that most of the data points are concentrated on the x-axis. 
This implies that most mainline have many dependent packages/projects compared to their variants counterparts.
However, we observe a few scattered data points a long the y-axes indicating a few variants which have many dependent packages/projects compared to their mainline counterparts. 
%We observe the same trend for the dependent projects in Figure~\ref{fig:packages_and_projects}~(b).
%In both Figure~\ref{fig:packages_and_projects}~(a) and Figure~\ref{fig:packages_and_projects}~(b), we observe that most variants have $<10$ dependent packages\,/\,projects. 
%This is still interesting since it implies that some developers do depend on the variants as opposed to their mainlines counterparts offering similar functionality. 
%In Figure~\ref{fig:mainline_variants_packages}-(a) we present a scatter plot showing the distribution of dependent projects of mainline vs the dependent packages of the mainline, while in Figure~\ref{fig:mainline_variants_packages}-(b) we show the same for variants. In both Figure~\ref{fig:mainline_variants_packages}-(a) and~\ref{fig:mainline_variants_packages}-(b) the y-axes represent the number of dependent projects for the mainlines\,/\,variants and the x-axes represent the number of dependent packages for the mainlines\,/\,variants, respectively. The color of the data points in the graphs represent the number cases for the mainlines\,/\,variants, respectively. We observe that the graph for the mainlines have more concentration of data points all over the graph compared the variants. this implies that the mainlines have more dependent packages/projects compared to the variants. However, irrespective of lower number of dependent packages/projects on the variants, it is still interesting to see some variants hanging dependents.

\begin{framed}
\noindent
\emph{Compared to the mainline counterparts, the variants have fewer dependent packages/projects. Since it is plausible to assume that the mainline and variants offer similar functionality because of the common code base, it is still interesting to observe that other projects use the variant package releases.}
\end{framed}
